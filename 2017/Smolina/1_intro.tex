\section*{Введение}
Графы и графовые базы данных имеют широкое применение в таких областях, как биоинформатика, логистика, социальные сети. Запросы к таким базам формулируются как задача поиска путей в графе, удовлетворяющих некоторым ограничениям. Во многих случаях такие ограничения формулируются в виде некоторой грамматики, например, контекстно-свободной (КС). В таком случае задача сводится к поиску путей в графе, которые бы соответствовали строкам контекстно-свободного языка. Такую задачу назовем синтаксическим анализом графа. 

Одним из примеров применения синтаксического анализа графов является поиск подпоследовательностей генома в метогеномной сборке. Путем поиска подпоследовательностей ДНК характерного вида осуществляется классификация организмов во взятом из окружающей среды образце, например, воды. Для этого по образцу строится метагеномная сборка, представляющая собой граф с последовательностями символов на ребрах. В таком графе необходимо найти подстроки, обладающие заданной вторичной структурой, специфицируемой формальной грамматикой. Такую задачу можно решить при помощи синтаксического анализа графов.

Существуют различные подходы к синтаксическому анализу графов (например,~\cite{GrigRagCFPQuerying},~\cite{Hellings120},~\cite{Sevon}), одно из которых основано на алгоритма Generalised LL (GLL)~\cite{GrigRagCFPQuerying}. Алгоритм GLL позволяет без модификаций грамматики анализировать все КС-языки. Результатом работы данного алгоритма является компактное представление леса разбора Shared Packed Parse Forest (SPPF)~\cite{SPPF}. Данное представление результата разбора позволяет производить дополнительный анализ, а также производить семантические действия после завершения синтаксического анализа. Данное решение построено по принципу генераторов синтаксических анализаторов, когда по грамматике генерируется код анализаторов, что не всегда удобно, например, при написании запросов к графовым базам данных.

Существуют различные решения для поиска путей в графовых базах данных. Это встроенные инструменты и языки для запросов. К примеру, для базы данных Neo4j~\cite{Neo4j} существуют такие языки запросов, как Cypher~\cite{Cypher} и openCypher~\cite{openCypher}. Однако они не поддерживают синтаксис запросов в стиле контекстно-свободных грамматик и не анализируют синтаксическую структуру путей. При работе с графовыми базами данных было бы удобно строить запрос к ним на языке, на котором написано целевое приложение. Это возможно реализовать техникой парсер-комбинаторов. Парсер-комбинаторы --- функции высшего порядка, которые либо возвращают анализаторы как результат, либо принимающие их в качестве аргументов. Все существующие библиотеки парсер-комбинаторов анализируют только линейный вход --- строки. Нашей задачей стала разработка библиотеки для синтаксического анализа графов. Существует библиотека Meerkat~\cite{Meerkat} на языке Scala~\cite{Scala}, реализующая синтаксический анализ строк методом парсер-комбинаторов, используя идеи, схожие с используемыми в алгоритме GLL~\cite{GLL}. Она обладает рядом преимуществ:

\begin{itemize} 
\item результатом работы библиотеки является лес разбора SPPF;
\item реализация в инфраструктуре JVM;
\item разбор происходит в худшем случае за $O(n^3)$, где n – длина входной последовательности.
\end{itemize}

Было принято решение использовать данную библиотеку для решения задачи.

Таким образом, была поставлена задача модифицировать существующую библиотеку парсер-комбинаторов для синтаксического анализа графов.
