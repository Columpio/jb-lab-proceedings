\section{Ограничения}

Предлагаемая реализация библиотеки сборки мусора использует функцию \lstinline[language= cpp]{mmap} для выделения памяти в
вируальном адресном пространстве. Подобное решение может использоваться лишь в UNIX-системах.
В ОС Windows существует функция \lstinline[language= cpp]{VirtualAlloc()}, предоставляющая требуемую функциональность.
Тем не менее, этого недостаточно для обеспечения переносимости сборщика мусора на ОС Windows.
Используемая в реализации куча Дага Ли, по словам самого автора\footnote{\cd{http://g.oswego.edu/dl/html/malloc.html}},
гарантирует правильность работы только на UNIX-подобных системах.
Проверки на совместимость с ОС Windows автором работы не проводилась.

Ещё одним уже упоминавшемся ограничением использования предлагаемой реализации является 64-х битная версия системы.
Как уже говорилось, работа сборщика мусора в 32-х битной системе возможна, но без предоставления возможности
совмещения ручного и автоматического управления памятью.