\section{Обзор}
Конечный преобразователь может быть задан следующей шестеркой элементов: $\langle Q, \Sigma, \Delta, q_0, F, E \rangle$, где

\begin{itemize}
\item $Q$ — множество состояний, 
\item $\Sigma$ — входной алфавит, 
\item $\Delta$ — выходной алфавит, 
\item $q_0 \in Q$ — начальное состояние, 
\item $F \subseteq Q$ — набор конечных состояний, 
\item $E \subseteq Q \times (\Sigma \cup \{\varepsilon\}) \times (\Delta \cup \{\varepsilon\})  \times Q$ — набор переходов. 
\end{itemize}

Композицией двух конечных преобразователей $T_1~=~\langle Q_1, \Sigma_1, \Delta_1, q_{0_{1}}, \\F_1, E_1 \rangle$ и $T_2~=~\langle Q_2, \Sigma_2, \Delta_2, q_{0_{2}}, F_2, E_2 \rangle$ является конечный преобразователь  $T =\langle Q_1  \times Q_2, \Sigma_1, \Delta_2, \langle q_{0_{1}}, q_{0_{2}} \rangle, F_1 \times F_2, E \cup E_{\varepsilon} \cup E_{i,\varepsilon} \cup E_{o,\varepsilon} \rangle$, где 

\begin{itemize}
\item $E = \{ \langle \langle p, q \rangle, a, b, \langle p', q' \rangle \rangle\ | \exists c \in \Delta_1 \cap \Sigma_2 : \langle p, a, c, p' \rangle \in E_1 \wedge \\\langle q, c, b, q' \rangle \in E_2\}$
\item $E_{\varepsilon} = \{ \langle \langle p, q \rangle, a, b, \langle p', q' \rangle \rangle\ | \langle p, a, {\varepsilon}, p' \rangle \in E_1 \wedge \langle q, {\varepsilon}, b, q' \rangle \in E_2\}$
\item $E_{i, \varepsilon} = \{ \langle \langle p, q \rangle, {\varepsilon}, a, \langle p, q' \rangle \rangle\ | \langle q, {\varepsilon}, a, q' \rangle \in E_2 \wedge p \in Q_1 \} $
\item $E_{o, \varepsilon} = \{ \langle \langle p, q \rangle,  a, {\varepsilon}, \langle p', q \rangle \rangle\ | \langle p, a, {\varepsilon}, p' \rangle \in E_1 \wedge q \in Q_2 \}. $
\end{itemize}

Текущая реализация алгоритма композиции FST в проекте\\ YaccConstructor работает по определению, т.е. перебирает всевозможные комбинации переходов FST участников композиции и добавляет в результирующий FST ребра, удовлетворяющие формальным описаниям из определения композиции. Поэтому временная сложность текущего алгоритма представляется как:

\[O(E_1 * E_2)\] где E — число ребер соответствующего FST.

Для удобства эту временную сложность можно представить в следующем виде:

\[O(V_1 * V_2 * D_1 * D_2)\] где V — число вершин, а D — максимальное число исходящих ребер соответствующего FST.

Важно заметить, что в результате работы алгоритма, несмотря на его корректность, в результирующем FST образуется множество недостижимых вершин, которые в дальнейшем с целью улучшения производительности, приходится удалять.

На замену текущему алгоритму был предложен алгоритм, представленный в статье \cite{handbook_automata}, потому что в результате его работы не образуется недостижимых вершин и он обладает лучшей асимптотической сложностью:

\[O(V_1 * V_2 * D_1 * (log(D_2) + M_2))\] где M — степень недетерминированности (максимальное количество исходящих из вершины ребер по одной метке).

Такая сложность достигается за счет поддержания очереди из пар вершин, в которых первая и вторая часть пары — вершины первого и второго FST соответственно. На каждой итерации алгоритма из очереди снимается одна пара вершин и перебираются комбинации ребер смежных этим вершинам. Далее, если метка выходного потока на ребре первого FST совпадает с меткой входного потока на ребре второго FST, то пара из вершин, в которые входят эти ребра, добавляется в очередь, а новое правило перехода добавляется в результирующий FST. Псевдокод алгоритма представлен в листинге (\ref{pseudocode}).

\begin{algorithm}[h]
 \caption{Композиция FST}
 \label{pseudocode}
 \begin{algorithmic}[1]
 \Procedure{Compose}{$I_1, I_2$}
 \State $Q\gets I_1\times I_2$
 \State $K\gets I_1\times I_2$ 
 \While{$K\neq \emptyset$}
   \State $q = (q_1, q_2)\gets Head(K)$
   \State $Dequeue(K)$
   \If{$q \in I_1\times I_2$}
     \State $I \gets I \bigcup \{q\}$
   \EndIf
   \If{$q \in F_1\times F_2$}
     \State $F \gets F \bigcup \{q\}$
   \EndIf
   \For{$\forall (e_1, e_2) \in E[q_1] \times E[q_2]: output[e_1] = input[e_2]$}
     \State $q' = (target[e_1], target[e_2])$
     \If{$q' \notin Q$}
       \State $Q\gets Q \bigcup \{q'\}$
       \State $Enqueue(K, q')$
     \EndIf
     \State $E\gets E \bigcup \{(q, input[e_1], output[e_2], q')\}$
   \EndFor
 \EndWhile  
 \State \textbf{return} $T$
 \EndProcedure
\end{algorithmic}
\end{algorithm}

Алгоритм возвращает новый FST $T$, множеством начальных вершин в котором является $I$, множеством конечных вершин — $F$, а множеством правил перехода — $E$. За основную очередь алгоритма обозначено $K$, а за $Q$ — множество посещенных вершин, которое помогает предотвратить зацикливание алгоритма на входах с циклами.

