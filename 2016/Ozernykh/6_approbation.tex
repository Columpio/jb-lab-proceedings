\section{Апробация}
\lstset{otherkeywords={if, fi, do, od, then, else}, basicstyle=\normalsize\ttfamily, columns=fullflexible, showspaces=false}
Апробация производится для языков While и Erlang. 
\subsection{Принтер для языка While}
Выбор данного языка был обусловлен тем, что количество структур в языке невелико  и что для него уже существовал принтер, заданный с использованием метода \cite{paper:while} и позволяющий форматировать программы на данном языке.
Грамматику потребовалось дополнить следующим образом.
В заголовок необходимо было добавить имя пакета для генерируемого принтера (\emph{printerPackage}), имя класса фабрики элементов (\emph{factoryClass}), имя класса внутреннего представления для файла данного языка (\emph{fileClass}).
Эти значения необходимо указывать, чтобы верно определить зависимости генерируемых классов.
Также в заголовке грамматики указывается список поддеревьев для файла данного языка (\emph{fileSubtrees}), о чем было упомянуто в разделе~\ref{sec:impl}.
В данном случае, fileSubtrees="procList, stmtList".
Далее необходимо было найти списочные правила и отметить их модификатором \emph{list}.
В языке While существует единственное списочное правило \lstinline[basicstyle=\normalsize\ttfamily, columns=fullflexible]{param_list}.
Таким образом, чтобы получить грамматику языка, потребовалось внести следующие изменения:
{
    \lstinputlisting[basicstyle=\normalsize\ttfamily, columns=fullflexible]{codes/whileGrammarDiff.txt}
}
\noindent
Полученный в результате работы генератора принтер\footnote{\texttt{https://github.com/prettyPrinting/whileLang-idea-plugin}} корректно форматировал программы на данном языке.
На рис.~\ref{app:whileEx} представлена программа на языке While, которую требуется отформатировать.
В качестве образцов форматирования будем использовать программы, представленные на рис.~\ref{app:whileTs}, \emph{a} и~\ref{app:whileTs}, \emph{б}.
Результат работы принтера представлен на рис.~\ref{app:whileRess}, \emph{a} и~\ref{app:whileRess}, \emph{б} соответственно.
%С помощью описанного в этой работе подхода удалось получить корректный принтер для программ на языке While.

\begin{figure}[h]
    \centering
    \lstinputlisting{codes/whileEx.txt}
    \caption{Пример кода на языка While}
    \label{app:whileEx}
\end{figure}


\begin{figure}[h]
  \noindent
  \begin{minipage}{.4\textwidth}
    \lstinputlisting{codes/whileT1.txt}
    \caption*{а)}    
  \end{minipage}
  \hfill
  \begin{minipage}{.5\textwidth}
    \lstinputlisting{codes/whileT2.txt}
    \caption*{б)}
  \end{minipage}
  \caption{Образцы форматирования}
  \label{app:whileTs}
\end{figure}

\begin{figure}[h]
  \noindent
  \begin{minipage}{.4\textwidth}
    \lstinputlisting{codes/whileRes1.txt}
    \caption*{а)}    
  \end{minipage}
  \hfill
  \begin{minipage}{.5\textwidth}
    \lstinputlisting{codes/whileRes2.txt}
    \caption*{б)}    
  \end{minipage}
  \caption{Результаты форматирования программы из рис.~\ref{app:whileEx}}
  \label{app:whileRess}
\end{figure}

\subsection{Принтер для языка Erlang}
Язык Erlang был выбран исходя из того, что для этого языка существовала грамматика, по которой генерировался синтаксический анализатор и классы внутреннего представления с помощью плагина Grammar-Kit.
Для генерации принтера по этой грамматики нужно было найти и отметить списочные правила, а также указать некоторую дополнительную информацию в заголовке файла, содержащего грамматику.

Однако полученный в результате реализованного генератора принтер некорректно форматировал некоторые структуры языка.
% связь
Во-первых, возникла проблема с форматированием некоторых списочных структур.
Связано это с тем, что в Erlang существуют списки, в которых одновременно могут быть различные типы разделителей.
Рассмотрим конструкцию сопоставления с образцом (pattern matching) для списков: [X,Y,Z|ZS], где X, Y, Z~-- элементы списка, ZS~-- его ``хвост''.
X, Y, Z, ZS с точки зрения синтаксического дерева являются выражениями, ``,'' и ``|''~-- разделителями.
Существующий способ форматирования поддерживает лишь единственный тип разделителей, поэтому при форматировании таких списков все разделители будут заменены на явно заданный или установленный по умолчанию разделитель (в данном случае ``|'' будет заменено на ``,''), что приведет к некорректности конструкции.

Во-вторых, некоторые правила грамматики данного языка недостаточно выразительны, что не влияет на синтаксический разбор выражений языка, но негативно отражается на их форматировании.
Например, бинарные операторы в грамматике Erlang задаются следующим образом:
\begin{lstlisting}[basicstyle=\normalsize\ttfamily, columns=fullflexible]
private add_op      ::= '+' |'-' | bor | bxor | bsl | bsr | or | xor
private mult_op     ::= '/' |'*' | div 0| rem  | band | and
\end{lstlisting}
\noindent
По данным правилам не генерируются классы внутреннего представления, а значит, по бинарному выражению невозможно узнать, какой именно оператор в нем задан (так как не генерируются соответствующие get-методы), то есть бинарный оператор не является поддеревом бинарных выражений.
Поэтому и принтер не будет строить представления для операторов, то есть выбирать тот, который соответствует данному узлу дерева и который требуется подставить в шаблон.
Таким образом, оператор будет явно ``зашит'' в шаблон для данной структуры: \lstinline[basicstyle=\normalsize\ttfamily, columns=fullflexible]{@expression + @expression}, где \lstinline[basicstyle=\normalsize\ttfamily, columns=fullflexible]{@expression}~-- место для вставки подвыражений.
Применяя такой шаблон к выражению с оператором ``-'', получится выражение с оператором ``+'', то есть семантика программы изменится, чего не должно происходить при переформатировании.
Чтобы решить эту проблему в грамматике языка Erlang, требуется полностью изменить способ задания бинарных выражений, что изменит классы внутреннего представления, а значит, может повлиять на работу плагина языка.

Однако некоторые структуры языка форматируются полученным принтером корректно. 
На рис.~\ref{app:erlT1} представлен образец кода на языке Erlang. Применяя шаблоны, полученные из этого образца к коду, представленному на рис.~\ref{app:erlExRes1}, \emph{a}, мы получаем код на рис.~\ref{app:erlExRes1}, \emph{б}.
\begin{figure}[h]
  \centering
  \begin{lstlisting}
foo(A, 0) -> A;
foo(A, B) -> foo(B, 0).
  \end{lstlisting}
  \caption{Образец кода на языке Erlang}
  \label{app:erlT1}
\end{figure}

\fvset{frame=lines,framesep=7pt}
\begin{figure}[h]
  \noindent
  \begin{minipage}{.45\textwidth}
    \begin{lstlisting}
b_not(true)->false;
b_not(false)->true.
b_and(true,true)->true;
b_and(_,_)->false.
b_or(false,false)->false;
b_or(_,_)->true.
    \end{lstlisting}
    \caption*{а) Неотформатированный код                }    
  \end{minipage}
  \hfill
  \begin{minipage}{.45\textwidth}
    \begin{lstlisting}
b_not(true) -> false; 
b_not(false) -> true.
b_and(true, true) -> true; 
b_and(_, _) -> false.
b_or(false, false) -> false;
b_or(_, _) -> true.
    \end{lstlisting}
    \caption*{б) Код с заданным форматированием}    
  \end{minipage}
  \caption{Пример форматирования программы на языке Erlang}
  \label{app:erlExRes1}
\end{figure}

\subsection{Итоги}
%Чтобы сгенерированный принтер был корректным, во-первых, требуется усовершенствовать метод \cite{paper:while}, добавив поддержку различных типов разделителей для списков, во-вторых, грамматику языка необходимо задавать в наиболее подробной форме, то есть, чтобы все элементы, которые имеют различные текстовые представления, имели свой класс во внутреннем представлении программы.
Апробация для языка Erlang показала, что не любая грамматика позволит получить принтер для языка, задаваемого этой грамматикой.
Во-первых, для структур языка, которые могут иметь различные текстовые представления, правила должны быть заданы таким образом, чтобы по ним генерировались классы внутреннего представления.
Например, чтобы избежать проблем с описанными выше бинарными выражениями, требуется, чтобы отдельно было задано правило для бинарных операций, что позволило бы рассматривать их как поддерево бинарных выражений, а не как терминальные символы.
Во-вторых, поддеревья структуры не должны пересекаться или быть вложенными друг в друга.
По умолчанию это условие выполняется, однако пользователь может также явно задавать поддеревья\footnote{\texttt{https://github.com/JetBrains/Grammar-Kit/blob/master/HOWTO.md\#32-organize-psi-using-fake-rules-and-user-methods}}.
В случае пересечения поддеревьев будет невозможно построить для них текстовые представления.


