\section{Постановка задачи}

Целью данной работы является автоматизация процесса задания декларативных принтеров путем их генерации по грамматике языка в форме Бэкуса-Наура.
Данная грамматика также используется для получения синтаксического анализатора программ на целевом языке с помощью плагина Grammar-Kit. 

Поставлены следующие задачи:
\begin{itemize}
    \item разработка методики генерации декларативных принтеров по грамматике в форме Бэкуса-Наура;
    \item интеграция метода в проект Grammar-Kit путем реализации генератора принтеров;
    \item реализация интеграции полученных принтеров с принтер-плагином для IDE IntelliJ IDEA;
    \item апробация метода на основе грамматики языков While \cite{paper:nielson} и Erlang.
\end{itemize}
While~--- учебный язык, содержащий самые основные конструкции.
На его примере производилась апробация метода генерации принтеров по языкозависимому описанию компонент \cite{paper:while}. 
Erlang\footnote{\texttt{https://www.erlang.org}}~--- промышленный язык программирования, для которого уже существует грамматика, по которой генерировался синтаксический анализатор с помощью плагина Grammar-Kit.
%по которой с помощью Grammar-Kit была реализована поддержка в среде IntelliJ IDEA.
%Использование именно этой грамматики предполагается для создания принтера языка Erlang.

% подумать