\section{Эксперименты}

Были проведены экспериментальные исследования, целью которых являлась проверка того, что конъюнктивные грамматики позволяют задавать структуру тРНК так, что синтаксический анализатор находит меньше некорректных цепочек.

\begin{figure}[h]
\begin{center}
\begin{verbatim}

[<Start>]
folded: stem<(any*[1..3] 
              stem<any*[7..10]> 
              any*[1..3] 
              stem<any*[5..8]> 
              any*[3..5] 
              stem<any*[5..8]>
              )>

stem<s>: 
      A stem<s> U
    | U stem<s> A
    | C stem<s> G
    | G stem<s> C
    | G stem<s> U
    | U stem<s> G
    | s

any: A | U | G | C

\end{verbatim}
\caption{КС-грамматика вторичной структуры тРНК}
\label{TRNAgrammar}
\end{center}
\end{figure}

\begin{figure}
\begin{center}
\begin{verbatim}
[<Start>]
folded: stem<subseq> & (any*[7..9] subseq any*[7..9])

subseq: any*[1..3] 
        stem<any*[7..10]> & (any*[4..6] any*[7..10] any*[4..6])
        any*[1..3] 
        stem<any*[5..8]> & (any*[6] any*[5..8] any*[6])
        any*[3..5] 
        stem<any*[5..8]> & (any*[4..5] any*[5..8] any*[4..5])
        
stem<s>:
      A stem<s> U
    | U stem<s> A
    | C stem<s> G
    | G stem<s> C
    | G stem<s> U
    | U stem<s> G
    | s

any: A | U | G | C

\end{verbatim}
\caption{Конъюнктивная грамматика структуры тРНК}
\label{TRNAgrammarConj}
\end{center}
\end{figure}


\begin{figure}
\begin{center}
\begin{tikzpicture}
\begin{axis}[
    legend pos = north west,
  xlabel = {Количество лексем},
  ylabel = {Время, с}
]
\addplot coordinates {
  (100,2) (200,17) (300,42) (400,81) (500,128) (600,190) (700,264) (800,345) (900,446) (1000,562)
};
\addplot coordinates {
  (100,1) (200,2) (300,4) (400,8) (500,12) (600,18) (700,22) (800,28) (900,36) (1000,43)
};
\legend{ 
  грамматика $G_{2}$, 
  грамматика $G_{3}$
};
\end{axis}
\end{tikzpicture}
\end{center}
\caption{Среднее время работы алгоритма на конъюнктивной и контекстно-свободной грамматиках тРНК}
\label{time}
\end{figure}

\begin{table}[h]
\begin{center}
  \begin{tabular}{ | c | c | c |}
    \hline
     & КС-грамматика & Конъюнктивная грамматика \\ \hline
    Тест 1. Кол-во ошибок: & 15 & 0 \\\hline
    Тест 2. Кол-во ошибок: & 5 & 0 \\\hline
    Тест 3. Кол-во ошибок: & 11 & 0 \\
    \hline
  \end{tabular}
\end{center}
\caption{Количество некорректных цепочек, распознанных синтаксическим анализатором}
\label{mistakes}
\end{table}

На рисунках~\ref{TRNAgrammar} и~\ref{TRNAgrammarConj} представлены грамматки, описывающие структуру тРНК. Грамматика $G_2$ является контекстно-свободной, а грамматика $G_3$ --- конъюнктивной. По данным грамматикам были сгенерированы соответствующие синтаксические анализаторы.

На вход построенным синтаксическим анализаторам подавались сгенерированные цепочки ДНК длиной от 100 до 1000 симовлов. Эти цепочки содержали в себе последовательности тРНК, а также другие последовательности, которые можно ложно признать за тРНК. Напимер, цепочка на рисунке~\ref{rnachain}, хоть и не является тРНК, распознаётся грамматикой $G_2$, но не распознаётся грамматикой $G_3$.

\begin{figure}
\begin{center}
ACACCCCCCCUCACCCCCUCCCACCCCCUU
\end{center}
\caption{Пример цепочки нуклеотидов, сгенеированной для экспериментов}
\label{rnachain}
\end{figure}


Результаты экспериментов приведены в таблице~\ref{mistakes}. Из них ясно, что грамматика $G_2$ не распознаёт ложные цепочки, распознавыемые грамматикой $G_3$. Время работы синтаксических анализаторов показано на графике, изображённом на рисунке~\ref{time}. По графику видно, что время работы синтаксического анализатора, построенного по грамматике $G_3$, значительно превышает время работы другого.

Таким образом, конъюнктивная грамматика позволяет отсеивать цепочки, ложно распозначаемые КС-грамматикой, но за время, значительно большее, чем время работы анализатора по КС-грамматике.
