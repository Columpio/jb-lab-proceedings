\section*{Введение}

Синтаксический анализ, как правило, используется для построения структурного представления кода с использованием грамматики, описывающей разбираемый язык. Абстрактное синтаксическое дерево, являющееся результатом работы синтаксического анализатора, в дальнейшем используется для проведения статического анализа кода или же в каких-то других целях. Как правило, на вход синтаксическому анализатору подаётся линейная последовательность токенов, представляющая код программы. Однако могут возникать ситуации, когда вход не может быть представлен линейно. Такие ситуации могут возникать, например, при автоматической генерации кода. Генерация может происходить в циклах, с использованием условных операторов или строковых операций. Поэтому для описания генерируемых цепочек можно использовать конечный автомат, порождающий цепочки, который уже не будет являться линейным. Такую задачу будем называть синтаксическим анализом регулярных множеств.

Кроме этого, ещё одной областью, где может быть применим синтаксический анализ регулярных множеств является бионформатика. Одной из часто возникающих задач в биоинформатике является классификация организмов, находящихся в образцах, полученных из окружающей среды~\cite{bioRNA}. По образцам строится метагеномная сборка, которая содержит в себе смесь из РНК всех содержащихся в сборке организмов. В свою очередь РНК является последовательностью символов в алфавите \{A, C, G, T\}. РНК организмов, которые относятся к одному и тому же виду, содержат одинаковые подцепочки, которые и необходимо выделить, чтобы классифицировать организм. Как правило, эти подцепочки --- это последовательности РНК. РНК может быть описана с помощью грамматики. Метагеномная сборка, в свою очередь, может быть представлена в виде графа с цепочками на рёбрах. Таким образом, в таком графе необходимо найти цепочки, выводимые в грамматике, описывающей РНК. 

Грамматики, описывающие структуру РНК, являются неоднозначными. Грамматика называется неоднозначной, если одна и та же цепочка может быть выведена несколькими способами. Такие алгоритмы синтаксического анализа как LR и LL не позволяют обрабатывать неоднозначные грамматики. Для работы с неоднозначными грамматиками существуют алгоритмы обобщённого синтаксического анализа GLR~\cite{GLR}, GLL~\cite{GLL}. В рамках проекта YaccConstructor~\cite{YaccConstructorPage, YaccConstructorPaper} был реализован алгоритм GLL, кроме того, была предложена его модификация для обработки нелинейных входных данных --- графов. Реализованная модификация позволяет находить цепочки транспортной РНК(тРНК) в небольших метагеномных сборках, возвращая координаты начала и конца найденной цепочки. Проблема заключается в том, что грамматика для описания тРНК является сильно неоднозначной, что сказывается на производительности и точности полученных результатов. Для повышения точности можно применять конъюнктивные грамматики~\cite{ConjGrammars}, в которых для описания продукций используется операция конъюнкции. Такие грамматики расширяют класс контекстно-свободных языков и позволяют точнее описать структуру тРНК. Данная работа посвящена описанию модификаций решения на основе алгоритма GLL для работы с конъюнктивными грамматиками.
