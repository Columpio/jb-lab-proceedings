\section*{Заключение}
В данной работе получены следующие результаты.
\begin{itemize}
\item Разработан алгоритм синтаксического анализа динамически  формируемого кода на основе алгоритма GLL, результатом работы которого является лес разбора, который компактно представляется с помощью структуры данных SPPF.
\item Доказана корректность и завершаемость предложенного алгоритма.
\item Предложенный алгоритм реализован на языке F\# в виде модуля инструмента YaccConstructor. Исходный код доступен в репозитории YaccConstructor~\cite{YCUrl}, автор работал под учётной записью \textit{AnastasiyaRagozina}.
\item Проведён ряд экспериментов и выполнено сравнение с алгоритмом, реализующим аналогичный подход.
\item Выполнена модификация предложенного алгоритма, позволяющая обрабатывать входные данные большого размера, что продемонстрировано на примере поиска подпоследовательностей в метагеномных сборках.
\end{itemize}

По результатам работы сделан доклад ``Обобщённый табличный LL-анализ'' на конференции ``ТМПА-2014'', тезисы опубликованы в сборнике материалов конференции,  и выполнена  публикация ``Средство разработки инструментов статического анализа встроенных языков'' в сборнике ``Наука и инновации в технических университетах материалы Восьмого Всероссийского форума студентов, аспирантов и молодых ученых''. Исследовательская работа поддержана грантом УМНИК: договор \textnumero 5609ГУ1/2014.

Существует несколько направлений дальнейшего развития полученных результатов. Во-первых, важной задачей является оценка теоретической сложности представленного алгоритма. Во-вторых, необходимо исследовать возможности по непосредственной поддержке грамматик в EBNF и поддержке булевых грамматик. Использование булевых, или даже конъюнктивных, грамматик позволит более точно задавать критерии поиска, например, это позволит специфицировать высоту \texttt{stem}-а. Эта возможность продемонстрирована в листинге~\ref{lst:conjExample}: правило \verb|stem_3_5<s>| описывает \texttt{stem} высотой от 3 до 5 пар.

%\fvset{frame=lines,framesep=5pt}
\begin{listing}
    \begin{pyglist}[language=ocaml,numbers=left,numbersep=5pt]

stem<s>: 
      A stem<s> U
    | U stem<s> A
    | C stem<s> G
    | G stem<s> C
    | G stem<s> U
    | U stem<s> G
    | s

any: A | U | G | C
stem_3_5<s>: stem <s> & (any*[3..5] s any*[3..5])

\end{pyglist}
\caption{Пример конъюнктивной грамматики для описания stem-ов фиксированной высоты}
\label{lst:conjExample}
\end{listing}


Кроме этого, необходимо выполнить ряд технических доработок, таких как оптимизация реализации.
