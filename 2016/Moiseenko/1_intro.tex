\section*{Введение}

C++ является популярным языком общего назначения, разработанным в 1980-х 
годах как улучшение широко используемого языка С. 
Главным новшеством языка было введение объектно-ориентированных конструкций, 
позволивших моделировать предметную область программы на языке объектов. 
С самого начала C++ разрабатывался с учётом высоких требований к производительности 
таким образом, чтобы введение новых языковых конструкций не нарушало обратную совместимость 
и не накладывало дополнительных накладых расходов (принцип ``don't pay for what you don't use''). 
C++ используется в приложениях, где критичными являются требования ко времени выполнения, 
потребляемой памяти, размеру исполняемых файлов. 
Среди таких приложений можно выделить финансовые и банковские системы (high-frequency trading), 
приложения, активно работающие с графикой (графические редакторы, системы виртуальной реальности, 
компьютерные игры), программы, работающие на встраиваемых платформах (embedded systems)\footnote{С обзором современного использования C++ можно ознакомиться здесь: 
\url{http://blog.jetbrains.com/clion/2015/07/infographics-cpp-facts-before-clion/}}.

Одним из основных ресурсов приложения является память. 
Большинство современных языков программирования активно использует \emph{динамическое 
распределение памяти}, при котором выделение памяти осуществляется во время исполнения программы. 
Динамическое управление памятью вводит два основных примитива~--- функции 
выделения и освобождения блоков памяти. 
Существуют два способа управления динамической памятью: \emph{ручное} и \emph{автоматическое}. 
При \emph{ручном управлении памятью} программист должен следить за освобождением выделенной 
памяти, что приводит к возможности возникновения труднообнаружимых ошибок. 
Более того, в некоторых ситуациях (например, при программировании на функциональных языках или 
в многопоточной среде) время жизни объекта не всегда очевидно для разработчика. 
\emph{Автоматическое управление памятью} избавляет программиста от необходимости вручную 
освобождать выделенную память, устраняя тем самым целый класс возможных ошибок и увеличивая 
безопасность исходного кода программы. 
\emph{Сборка мусора} (garbage collection) давно стала стандартом в области автоматического 
управления памятью, хотя её использование может накладывать дополнительные расходы по памяти 
и времени исполнения.

На сегодняшний день среды времени выполнения многих популярных языков программирования, 
таких как Java, C\#, Python, Ruby и др., активно используют сборку мусора. 
Язык C++ разрабатывался с расчётом на использование ручного управления памятью. 
Некоторые языковые возможности, такие как адресная арифметика и приведение типов указателей, 
затрудняют реализацию сборщиков мусора для этого языка. 
Несмотря на это, существует ряд подходов к разработке сборщиков мусора для языка C++. 

В рамках проекта лаборатории JetBrains была реализована сборка мусора для языка 
C++~\cite{book:precisegc_berezun,book:precisegc_samofalov,book:precisegc_secr} на уровне 
библиотеки. 
Сборщик приостанавливает работу приложения на всё время сборки мусора, из-за чего в работе 
программы могут возникать длительные паузы. 
Продолжительные остановки на сборку мусора неприемлимы для целого класса приложений, 
для которых важна низкая латентность, то есть малое время отклика на запросы пользователей. 
Существуют подходы, призванные уменьшить время паузы на сборку мусора, 
в частности, инкрементальная параллельная маркировка. 
Отличительной особенностью данного подхода является выполнение части работы по 
сборке мусора параллельно с приложением, без его полной остановки.
