\section{Реализация}

В данном разделе приводится описание выполненных модификаций библиотеки
и деталей их реализации.


\subsection{Остановка мира}
\label{sec:stw}
В разделе~\ref{sec:old_lib} были перечислены недостатки использовавшегося алгоритма 
остановки мутатора. 
Для их преодоления был предложен новый подход. 
Вместо расстановки безопасных точек в примитивах библиотеки предлагается помечать 
небезопасные участки кода, а другие участки считать безопасными. 
Преимущество данного подхода в том, что теперь мутатор может быть уведомлён 
об инициации сборки мусора в любой момент своей работы. 
Если в момент инициации сборки поток находится в небезопасном участке, то его приостановка 
откладывается до момента, когда он покинет этот участок. 
Заметим, что небезопасными являются только участки кода, выполняющие некоторые операции с 
глобальными структурами данных (корневым множеством, кучей, и т.д.) и расположенные в 
примитивах самой библиотеки.

Для реализации этого подхода необходимо иметь возможность уведомить поток об инициации 
сборки мусора в любой момент времени. 
После получения такого уведомления поток должен приостановить свою работу до момента, 
когда сборка будет окончена. 
Подобный функционал может быть реализован при помощи механизма сигналов операционной системы 
\code{Linux} и примитивов синхронизации, являющихся 
асинхронно-сигналобезопасными. 
Далее в данной главе будут рассмотрены сигналы и реализация механизма их временной блокировки, 
учитывающая специфику их использования в нашей библиотеке. 
Также будет описана реализация асинхронно-сигналобезопасных примитивов синхронизации. 
В последнем подразделе описан новый алгоритм ``остановки мира''. 


\subsubsection{Сигналы ОС \code{Linux}}
Сигналы в \code{Linux}~---~один из механизмов межпроцессорного взаимодействия \cite{book:os_linux}. 
\code{Linux} поддерживает несколько типов сигналов, каждый из которых имеет уникальный идентификатор 
и мнемоническое имя. 
Сигнал может быть отправлен процессу либо ядром операционной системы, либо другим процессом 
с помощью системного вызова \code{kill}. 
При получении сигнала вызывается соответствующий обработчик сигнала. 
Для некоторых типов сигналов разрешается переопределять обработчик сигнала (при помощи 
системного вызова \code{sigaction}). 
Также имеется возможность временно заблокировать доставку сигналов определенного типа 
(системный вызов \code{sigprocmask}).

Библиотека \code{pthreads} позволяет отправлять сигналы конкретному потоку процесса с 
помощью системного вызова \code{pthread\_kill}, тем самым позволяя использовать сигналы 
для межпоточного взаимодействия.


\subsubsection{Блокирование сигналов}
В библиотеке \code{pthreads} имеется функция \code{pthread\_sigmask} аналогичная 
\code{sigprocmask}, но блокирующая доставку сигналов только данному потоку. 
Однако при каждом вызове \code{pthread\_sigmask} происходит переход в \emph{режим ядра} 
(\emph{kernel mode}), что вызывет длительные задержки. 
Временное блокирование доставки сигнала об инициации сборки мусора очень частая операция. 
Например, она необходима при каждой модификации корневого множества. 
Из-за её низкой скорости выполнения может пострадать производительность всего приложения, 
используещего сборку мусора. 

Для того, чтобы ускорить блокирование доставки сигнала инициации сборки мусора, был 
реализован механизм, учитывающий специфику задачи. 
Идея заключается в том, чтобы отследить попытку доставки сигнала потоку, выполнившему его 
блокирование, и отложить выполнение обработчика до момента разблокировки. 
В каждом потоке объявляются две локальные для потока (\code{thread\_local}) переменные 
\code{depth} и \code{pending\_flag}. \code{depth} отслеживает вложенность блокировок, 
\code{pending\_flag} инициализируется значением \code{false}. 
В обработчике сигнала сначала проверяется значение \code{depth}. 
Если оно не равно нулю, значит в текущий момент доставка сигнала заблокирована, 
выставляется \code{pending\_flag}, и происходит выход из обработчика. 
Иначе происходит нормальное исполнение обработчика сигнала. 
Для того чтобы заблокировать доставку сигнала поток просто инкрементирует переменную 
\code{depth}. 
При разблокировании доставки сигнала значение \code{depth} декрементируется, и если оно 
становится равным нулю, и при этом установлен флаг \code{pending\_flag}, вызывается код 
обработчика сигнала, но уже без дополнительных проверок. 
Код обработчика сигнала, блокирования и разблокирования доставки сигналов на C++ может быть 
найден в листинге~\ref{code:siglock}.

В таблице~\ref{table:siglock} приводится сравнение времени работы механизма блокирования 
сигналов с помощью функции \code{pthread\_sigmask} и механизма, предложенного в нашей работе. 
Измерялось суммарное время работы пары операций блокирования и разблокирования сигнала. 
В таблице приведено среднее время работы операций $\bar{x}$ и стандартное отклонение $s$. 
Заметим, что ускорения удалось достичь за счёт того, что предложенный подход к блокированию 
сигналов является менее общим и учитывает специфику нашей задачи.

\begin{table}[h!]
\caption{Сравнение времени работы двух методов блокирования сигналов (среднее время $\bar{x}$ 
и стандартное отклонение $s$ указано в наносекундах)}
\label{table:siglock}
\begin{center}
\begin{tabular}{|l|c|c|}
\hline
                            & $\bar{x}$    & $s$       \\ \hline
pthread\_sigmask            & 143.661      & 6.741     \\ \hline
siglock/sigunlock           & 3.965        & 0.033     \\ \hline
\end{tabular}
\end{center}
\end{table}


\subsubsection{Асинхронно-сигналобезопасные примитивы синхронизации}
Стандарт POSIX\footnote{Стандарт POSIX описывает интерфейс для доступа к функциям 
операционной системы. 
Обеспечивает переносимость прикладных программ между ОС, поддерживающими POSIX. \\
\url{http://pubs.opengroup.org/onlinepubs/9699919799/}} 
накладывает жёсткие ограничения на код, работающий внутри обработчика сигнала. 
Обработчик сигнала должен быть асинхронно-сигналобезопасной функцией 
\footnote{Определение асинхронно-сигналобезопасной функции 
может быть найдено по ссылке\\ \url{https://www.securecoding.cert.org/confluence/display/c/BB.+Defi}\\{nitions\#BB.Definitions-asynchronous-safefunction}}
(\emph{async-signal safe function}). 

После получения сигнала об инициации сборки мусора поток должен приостановить свою работу 
до тех пор, пока он не получит уведомление об окончании сбоки. 
Код, выполняющий эту задачу, должен работать в контексте обработчика сигнала, так как иного 
способа уведомить поток о некотором событии в произвольный момент времени, кроме использования сигналов, нет. 
Так как никакие из примитивов синхронизации библиотеки \code{pthreads} не являются 
асинхронно-сигналобезопасными\footnote{Список async-signal safe функций ОС \code{Linux} может 
быть найден по ссылке\\ \url{http://pubs.opengroup.org/onlinepubs/9699919799/functions/
V2\_chap02.html\#tag\_15\_04\_03}}, 
необходимо использовать собственную реализацию примитивов синхронизации.

В нашей библиотеке было реализована два примитива: \code{signal\_safe\_barrier} и 
\code{signal\_safe\_event}. 

\begin{itemize}
\item 
    \code{signal\_safe\_barrier} имеет два метода:
    \begin{enumerate}
    \item 
        \code{notify()} для уведомления о том, что поток достиг некоторой точки в программе;
    \item 
        \code{wait(size\_t n)} для приостановки текущего потока до момента, когда \code{n} 
        других потоков достигнут барьера.
    \end{enumerate}
\item 
    \code{signal\_safe\_event} также определяет два метода:
    \begin{enumerate}
    \item 
        \code{notify(size\_t n)} для уведомления \code{n} потоков о наступлении события;
    \item 
        \code{wait()} для ожидания события.
    \end{enumerate}
\end{itemize}

Оба примитива реализованы с помощью одной и той же техники. 
Системные вызовы \code{read} и \code{write}, предназначенные для чтения и записи данных из 
потока ввода-ввывода, являются асинхронно-сигналобезопасными. 
Кроме того, \code{read} является блокирующим, то есть поток, вызвавший \code{read}, 
будет приостановлен, пока из потока не будет прочитано указанное количество байт. 
На основе этих двух системных вызовов, а также системного вызова \code{pipe}, 
открывающего безымянный двунаправленный поток ввода-ввывода, и построены наши примитивы. 

Оба примитива реализованы как классы. 
В конструкторе они открывают безымянный поток ввода-вывода. 
Метод \code{notify()} класса \code{signal\_safe\_barrier} записывает один байт в этот поток. 
Метод \code{wait(size\_t n)} читает из потока \code{n} байт. 
Аналогично для \code{signal\_safe\_event}, \code{notify(size\_t n)} записывает \code{n} байт 
в поток, а \code{wait()} читает из потока один байт.

\begin{figure}
\begin{lstlisting}[language={c++}, caption={Блокирование доставки сигналов}, label={code:siglock}, basicstyle=\small]
thread_local volatile sig_atomic_t depth = 0;
thread_local volatile sig_atomic_t pending_flag = false;

void check_gc_siglock(int signum) {
    if (depth > 0) {
        pending_flag = true;
        return;
    }
    gc_signal_handler();
}

void siglock() {
    if (depth == 0) {
        std::atomic_signal_fence();
        depth = 1;
        std::atomic_signal_fence();
    } else {
        depth++;
    }
}

void sigunlock() {
    if (depth == 1) {
        std::atomic_signal_fence();
        depth = 0;
        std::atomic_signal_fence();
        bool pending = pending_flag;
        pending_flag = false;
        if (pending) {
            gc_signal_handler();
        }
    } else {
        depth--;
    }
}
\end{lstlisting}
\end{figure}


\subsubsection{Алгоритм ``остановки мира''}
Опишем теперь сам алгоритм ``остановки мира''. 
Рассмотрим отдельно часть, отвечающую за приостановку потока, выполняющуюся в 
обработчике сигнала, и часть, инициирующую сборку. 
В ходе выполнения обе части синхронизируют свою работу с помощью одного объекта класса 
\code{signal\_safe\_barrier} и одного объекта \code{signal\_safe\_event}. 
В обработчике сигнала поток сначала вызывает метод \code{notify()} барьера, чтобы 
уведомить поток, иницировавший ``остановку мира'', что он достиг обработчика. 
Затем вызывается метод \code{wait()} события. 
Когда сборка мусора окончится, поток, выполнявший сборку, возобновит работу остальных 
потоков, вызвав метод \code{notify(size\_t)} этого события. 
Затем в обработчике события будет вызван \code{notify()} барьера ещё раз, чтобы 
уведомить поток инициатор, что обработка сигнала завершена, и поток возвращается к 
нормальному исполнению. 
Для того, чтобы два потока одновременно не инициировали ``остановку мира'', код инициирования 
защищён мьютексом. 
Псевдокод иницирования ``остановки мира'' приведён в листинге~\ref{code:stw}.

\begin{figure}[h!]
\begin{lstlisting}[caption={Остановка мира},label={code:stw}]
lock(gc_mutex)
for (thread in threads)
    send(gc_signal, thread)
signal_safe_barrier.wait(threads.count - 1)
gc()
signal_safe_event.notify(threads.count - 1)
signal_safe_barrier.wait(threads.count - 1)
unlock(gc_mutex)
\end{lstlisting}
\end{figure}


\subsection{Закрепление объектов}
Механизм закрепления объектов, использовавшийся в предыдущей версии библиотеки 
(раздел~\ref{sec:old_lib}), включал процедуру консервативного обхода стека потоков. 
Поэтому в некоторых ситуациях сборщик мог ошибочно не идентифицировать мусор. 
В новой версии реализован механизм закрепления объектов, лишённый этого недостатка, и 
кроме того, точно отслеживающий время, когда закрепление может быть снято.

Используется идиома RAII, упомянутая в разделе~\ref{sec:cpp_solutions}. 
Вводится новый примитив~--- шаблонный класс \code{gc\_pin<T>}. 
Конструктор класса \code{gc\_pin<T>} принимает в качестве аргумента ссылку на объект типа 
\code{gc\_ptr<T>} и выполняет закрепление объекта, на который указывает этот \code{gc\_ptr}. 
В деструкторе \code{gc\_pin<T>} закрепление объекта снимается. 
Указатели на все закрепленные объекты хранятся в структуре данных, аналогичной той, 
что используется для хранения корневого множества (раздел~\ref{sec:old_lib}). 
То есть используется список указателей, причём новые указатели вставляются в голову и 
при удалении список просматривается с головы. 
Также как и в случае корневого множества, каждый поток имеет собственный экземпляр списка 
закрепленных объектов. 
Таким образом, в конструкторе \code{gc\_pin<T>} указатель на объект вносится в список 
закрепленных объектов, а в деструкторе удаляется из него.

На этапе сжатия/освобождения кучи, после ``остановки мира'', сборщик просматривает списки 
закрепленных объектов всех потоков. 
У объектов из этих списков выставляется бит закрепления (наличие этого бита говорит сборщику 
о запрете перемещать данный объект), также они сканируются сборщиком (для того, чтобы 
пометить все объекты, достижимые из закрепленного, и предотвратить их удаление). 

Как уже упоминалось, закрепление объектов необходимо в двух случаях:
\begin{enumerate}
\item 
    Для передачи сырого указателя на управляемый объект в функцию, 
    принимающую сырой указатель как аргумент.
\item 
    Для закрепления указателя \code{this} при вызове оператора доступа к члену класса 
    (\code{operator->()}).
\end{enumerate}

В первом случае пользователь может самостоятельно создать объект типа \code{gc\_pin} и 
получить сырой указатель с помощью метода \code{get()} этого объекта. 
Стоит однако отметить, что время жизни полученного таким образом сырого указателя не 
должно превышать время жизни соответствующего объекта \code{gc\_pin}. 
Во втором случае объект \code{gc\_pin} создаётся неявно самой библиотекой. 
Используется следующее соглашения языка C++ 
\footnote{Working Draft, Standard for Programming Language C++,\\
\url{http://www.open-std.org/jtc1/sc22/wg21/docs/papers/2014/n4296.}\\{pdf}}
: если оператор доступа к 
члену класса (\code{operator->()}) возвращает объект типа, отличного от \code{T*}, 
вызывается оператор доступа к члену класса этого объекта. 
То есть, перегруженный оператор доступа к члену класса \code{gc\_ptr} создаёт на стеке 
объект типа \code{gc\_pin} и возвращает его, затем, согласно соглашению языка C++, 
вызывается оператор доступа к члену класса этого временного объекта, 
который уже возвращает сырой указатель. 
После этого вызывается деструктор временного объекта \code{gc\_pin}.  


\subsection{Инкрементальная параллельная маркировка}
Рассмотрим реализацию алгоритма инкрементальной параллельной маркировки, разработанную в 
рамках данной работы и внедренную в библиотеку точной сборки мусора для языка C++.

В нашей работе в фазе маркировки работа мутатора не приостанавливается, сборщик мусора 
работает параллельно с потоками мутатора. 
``Остановка мира'' происходит два раза за цикл сборки мусора: перед запуском этапа 
маркировки для сканирования корневого множества и для освобождения памяти после 
окончания маркировки. 
Выполнять маркировку параллельно могут сразу несколько потоков. 
Фаза маркировки начинается, когда размер кучи достигает некоторого порогового значения. 

Для распределения работы между потоками-маркировщиками используется механизм 
\emph{пакетов работы} (\emph{work packet}), подобный описанному в статье 
\cite{barabash2005parallel}. 
Пакет представляет собой массив фиксированного размера, который содержит указатели на 
объекты, ожидающие сканирования. 
Сборщик мусора поддерживает глобальный пул пакетов. 
Каждый поток--маркировщик использует два пакета~---~входной (input) и выходной (output). 
Маркировщик по очереди вынимает указатели из входного пакета и сканирует объекты, на которые 
они указывают. 
Если обнаружится, что объект содержит указатели на другие объекты, они заносятся в выходной 
пакет. 
Когда входной пакет оказывается пуст или выходной пакет заполнен, маркировщик возвращает пакет 
в глобальный пул и получает новый пакет. 

Глобальный пул пакетов поддерживает три списка: список пустых пакетов, 
список частично (менее чем на 50\%) заполненных пакетов и список почти 
полностью (более чем на 50\%) заполненных пакетов. 
Каждый список также содержит мьютекс, для того, чтобы синхронизировать операции добавления 
и удаления элементов при обращении нескольких потоков. 
Суммарное количество пакетов фиксированно, память для пакетов выделяется один раз 
при инициализации сборщика мусора. 
Когда поток--маркировщик запрашивает входной пакет у пула, выбирается пакет из почти 
полностью заполненных, а если таковых нет~---~из частично заполненных. 
При запросе выходного пакета возвращается пустой пакет, если же список пустых пакетов пуст, 
то пакет из частично заполненных. 
Изначально все пакеты находятся в списке пустых пакетов. 
При сканированни корневого множества сборщик запрашивает у пула необходимое количество 
выходных пакетов, копирует в них указатели на корневые объекты и возвращает пакеты в пул. 
После этого потоки-маркировщики начинают запрашивать входные пакеты и сканировать их. 

Использование механизма пакетов позволяет достаточно просто определить момент окончания 
фазы маркировки~--- когда количество пустых пакетов становится равным общему количеству 
пакетов, фаза маркировки считается законченной. 

Рассмотрим также реализацию барьера записи. 
В нашей библиотеке барьер записи реализован программно и вызывается из конструктора 
копирования и оператора присваивания \code{gc\_ptr}. 
Барьер записи перехватывает операции записи только если сборщик находится в фазе маркировки. 
Мы используем барьер записи Дейкстры, его псевдокод приведён в листинге~\ref{code:write_barrier}. 
Напомним, что барьер записи Дейкстры перекрашивает объект, ссылка на который записывается, 
в серый цвет, если раньше его цвет был белым. 
Для этого каждый поток мутатора запрашивает у глобального пула пакетов один выходной пакет 
и помещает в него ссылки на ``перекрашенные'' в серый объекты. 
Когда пакет заполняется, поток возвращает его в пул и запрашивает новый. 
Заметим, что после окончания фазы маркировки у потоков мутатора могут остаться частично 
заполненные пакеты серых объектов. 
Данные пакеты сканируются уже после ``остановки мира''. 
Ожидается, что количество объектов в этих частично заполненных пакетах будет небольшим.

Также стоит отметить, что в фазе маркировки новые объекты создаются чёрными, 
то есть гарантированно переживают текущий цикл сборки мусора.

Операции присваивания указателей и создания новых объектов очень часты, 
поэтому необходимо, чтобы проверка текущей фазы была достаточно быстрой. 
В нашей реализации для этого используется переменная \code{phase} типа 
перечисления (допустимые значения: \code{IDLE}, \code{MARKING}, \code{COMPACTING}). 
Изменения значения переменной \code{phase} происходят только во время остановок мира, 
а операции присваивания указателей и создания нового объекта помечены как небезопасные, 
то есть во время их выполнения не может произойти ``остановки мира'' и, следовательно, 
не может измениться фаза сборки мусора.


\subsection{Параллельное сжатие}
В процедуру сжатия кучи также были внесены изменения с целью её распараллеливания. 
Как уже упоминалось в разделе~\ref{sec:heap}, куча для объектов маленьких размеров 
представляет собой набор пулов, каждый из которых обслуживает объекты определенного размера. 
Ясно, что каждый пул можно сжимать независимо от остальных в отдельном потоке. 
В нашей реализации множество пулов разбивается на $n$ равных частей, 
где $n$~--- количество доступных на данной системе ядер процессора, 
полученное с помощью функции из стандартной библиотеки 
\code{std::thread::hardware\_concurrency}. 
Затем создаётся $n$ потоков, каждый из которых поочередно сжимает пулы из 
соответствующей части с помощью двухпальцевого алгоритма~\ref{sec:two-finger-compact}. 
Преимущество подобного статического разбиения работ в том, что потокам нет необходимости 
синхронизировать свою работу. 
С другой стороны, если окажется, что объекты распределены по пулам неравномерно, 
то сборщик не сможет полностью задействовать аппаратный параллелизм, так как некторые 
потоки закончат свою работу намного раньше остальных. 

Объекты большого размера выделяются с помощью системного вызова \code{mmap} и 
сохраняются в список больших объектов. 
Для таких объектов сжатие не выполняется. 
Память из-под объектов, объявленных сборщиком недоступными, возвращается операционной 
системе с помощью системного вызова \code{munmap}.
