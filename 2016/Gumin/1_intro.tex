\section*{Введение}
Существует подход к написанию программного кода, при котором код на одном языке программирования формируется программой на другом языке с помощью строковых операций, циклов и условных операторов. Такой код называется динамически формируемым. В качестве примера можно рассмотреть генерацию html-страниц в языке JavaScript или формирования SQL-запросов в C\# с помощью строковых операций. На листинге~\ref{lst:example1} представлен пример динамически формируемого кода.

\fvset{frame=lines,framesep=5pt}
\begin{listing}[H]
    \begin{pyglist}[language=csharp,numbers=left,numbersep=5pt]
 private void Go(int cond){
   string columnName = cond > 3 ? "X" : (cond < 0 ? "Y" : "Z");
   string queryString = 
       "SELECT name" + columnName + " FROM table";
   Program.ExecuteImmediate(queryString);}
    \end{pyglist}
\caption{Пример динамически формируемого кода}
\label{lst:example1}
\end{listing}

Этот подход был достаточно распространен, поэтому существует большое множество программ, использующих его. Кроме того, иногда возникает необходимость написания такого кода (например, в случаях, когда ограничения по производительности не позволяют использовать ORM). Статический анализ такого кода, например, подсветка синтаксиса и диагностика ошибок, позволил бы существенно упростить его написание и поддержку.

Существует ряд инструментов~\cite{Alvor, JSA, PHPSA} (подробно рассмотрены в работе~\cite{polubelova}), позволяющих проводить статический анализ динамически формируемого кода. Однако эти инструменты обладают такими недостатками, как слабая расширяемость, ограниченная функциональность, применимость только к конкретным языкам. Для решения этих проблем~\cite{GrigorievPhd} в рамках проекта YaccConstructor~\cite{YCUrl, articleYC}, который посвящен исследованиям в области лексического и синтаксического анализа, реализована платформа для работы со встроенными языками. Модульность платформы позволяет легко добавлять дополнительную функциональность и поддержку новых языков. Статический анализ динамического кода производится в несколько шагов: 
\begin{enumerate}
\item построение регулярной аппроксимации сверху множества значений динамически формируемого выражения;
\item лексический анализ;
\item синтаксический анализ;
\item семантический анализ.
\end{enumerate}

Но механизм лексического анализа~\cite{polubelova} в проекте YaccConstructor, использующий композицию конечных преобразователей, обладает недостаточной производительностью. Вероятная причина этой проблемы~--- разрастание конечных преобразователей на больших алфавитах, так как обработка большого количества дуг занимает много времени. Возможное решение~--- применение символьных конечных преобразователей~\cite{st}, с помощью которых можно представить структуры данных более компактно. 
На данный момент единственной библиотекой для работы с символьными конечными преобразователями в рамках платформы .NET является библиотека Microsoft.Automata~\cite{MSAUrl}, которая активно поддерживается сообществом Microsoft Research, именно поэтому она и является предметом изучения.
В рамках данной работы исследован вопрос о возможности увеличения производительности лексического анализа с помощью использования символьных конечных преобразователей из библиотеки Microsoft.Automata.
