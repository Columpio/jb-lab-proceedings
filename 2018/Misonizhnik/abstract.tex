\begin{abstract}
%
Номинальные системы типов с вариантностью, составляющие основу отношения подтипирования в объектно-ориентированных языках таких как Java, C\# и Scala, были подробно изучены Кеннеди и Пирсом. Они доказали неразрешимость отношения подтипирования между закрытыми типами в общем случае и предъявили несколько разрешимых фрагментов задачи.
Однако, для точного межпроцедурного анализа объектно-ориентированных языков могут потребоваться рассуждения об отношении подтипирования между открытыми типами. В этой статье мы формализовали и исследовали задачу выполнимости ограничений номинальной системы типов с вариантностью. Мы определили проблемы в терминах логики первого порядка. Мы показали, что даже для нерасширяющихся таблиц классов, для которых разрешимо отношение подтипирования между закрытыми типами, задача выполнимости неразрешима. В нашем доказательстве используется удивительно маленький фрагмент системы типов. Фактически мы показали, что даже для нерасширяющихся таблиц классов, содержащих только нульарные и унарные типовые конструкторы, задача выполнимости бескванторной формулы, состоящей из конъюнкции атомов подтипирования без отрицаний, неразрешима.
%
\end{abstract}
