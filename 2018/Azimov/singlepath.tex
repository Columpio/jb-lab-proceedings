\section{CFPQ using single-path query semantics}
In this section, we show how the context-free path query evaluation using the single-path query semantics can be reduced to the calculation of matrix transitive closure $a^{cf}$ and prove the correctness of this reduction.

At the first step, we show how the calculation of matrix transitive closure $a^{cf}$ which was discussed in Section~\ref{section_reducing} can be modified to compute the length of some path $i \pi j$ for all $(i,j) \in R_A$, such that $A \xrightarrow{*} l(\pi)$. This is sufficient to solve the problem of context-free path query evaluation using the single-path query semantics since the required path of a fixed length from the node $i$ to the node $j$ can be found by a simple search and checking whether the labels of this path form a string which can be derived from a non-terminal $A$.

Let $G = (N,\Sigma,P)$ be a grammar and $D = (V, E)$ be a graph. We enumerate the nodes of the graph $D$ from 0 to $(|V| - 1)$. We initialize the $|V| \times |V|$ matrix $a$ with $\varnothing$. We associate each non-terminal in matrix $a$ with the corresponding path length. For convenience, each nonterminal $A$ in the $a_{i,j}$ is represented as a pair $(A,k)$ where $k$ is an associated path length. For every $i$ and $j$ we set $$a_{i,j} = \{(A_k,1)~|~((i,x,j) \in E) \wedge ((A_k \rightarrow x) \in P)\}$$ since initially all path lengths are equal to $1$. Finally, we compute the transitive closure $a^{cf}$ and if non-terminal $A$ is added to $a^{(p)}_{i,j}$ by using the production rule $(A \rightarrow B C) \in P$ where $(B,l_B) \in a^{(p-1)}_{i,k}$, $(C,l_C) \in a^{(p-1)}_{k,j}$, then the path length $l_A$ associated with non-terminal $A$ is calculated as $l_A = l_B + l_C$. Therefore $(A, l_A) \in a^{(p)}_{i,j}$. Note that if some non-terminal $A$ with an associated path length $l_1$ is in $a^{(p)}_{i,j}$, then the non-terminal $A$ is not added to the $a^{(k)}_{i,j}$ with an associated path length $l_2$ for all $l_2 \neq l_1$ and $k \geq p$. For the transitive closure $a^{cf}$, the following statements hold.

\begin{lemma}\label{lemma:singlepath}
	Let $D = (V,E)$ be a graph, let $G =(N,\Sigma,P)$ be a grammar. Then for any $i, j$ and for any non-terminal $A \in N$, if $(A,l_A) \in a^{(k)}_{i,j}$, then there is $i \pi j$, such that $A \xrightarrow{*} l(\pi)$ and the length of $\pi$ is equal to $l_A$.
\end{lemma}
\begin{proof}(Proof by Induction)
	
	\textbf{Basis}: Show that the statement of the lemma holds for $k = 1$. For any $i, j$ and for any non-terminal $A \in N$, $(A, l_A) \in a^{(1)}_{i,j}$ iff $l_A = 1$ and there is $i \pi j$ that consists of a unique edge $e$ from the node $i$ to the node $j$ and $(A \rightarrow x) \in P$ where $x = l(\pi)$. Therefore there is $i \pi j$, such that $A \xrightarrow{*} l(\pi)$ and the length of $\pi$ is equal to $l_A$. Thus, it has been shown that the statement of the lemma holds for $k = 1$.
	
	\textbf{Inductive step}: Assume that the statement of the lemma holds for any $k \leq (p - 1)$ and show that it also holds for $k = p$ where $p \geq 2$. For any $i, j$ and for any non-terminal $A \in N$, $(A, l_A) \in a^{(p)}_{i,j}$ iff $(A, l_A) \in a^{(p-1)}_{i,j}$ or $(A, l_A) \in (a^{(p-1)} \times a^{(p-1)})_{i,j}$ since $a^{(p)} = a^{(p-1)} \cup (a^{(p-1)} \times a^{(p-1)}).$
	
	Let $(A, l_A) \in a^{(p-1)}_{i,j}$. By the inductive hypothesis, there is $i \pi j$, such that $A \xrightarrow{*} l(\pi)$ and the length of $\pi$ is equal to $l_A$. Therefore the statement of the lemma holds for $k = p$.
	
	Let $(A, l_A) \in (a^{(p-1)} \times a^{(p-1)})_{i,j}$. By the definition, $(A, l_A) \in (a^{(p-1)} \times a^{(p-1)})_{i,j}$ iff there are $r$, $(B, l_B) \in a^{(p-1)}_{i,r}$ and $(C, l_C) \in a^{(p-1)}_{r,j}$, such that $(A \rightarrow B C) \in P$ and $l_A = l_B + l_C$. Hence, by the inductive hypothesis, there are $i \pi_1 r$ and $r \pi_2 j$, such that $$(B \xrightarrow{*} l(\pi_1)) \wedge(C \xrightarrow{*} l(\pi_2)),$$ where the length of $\pi_1$ is equal to $l_B$ and the length of $\pi_2$ is equal to $l_C$. Thus, the concatenation of paths $\pi_1$ and $\pi_2$ is $i \pi j$, where $A \xrightarrow{*} l(\pi)$ and the length of $\pi$ is equal to $l_A$. Therefore the statement of the lemma holds for $k = p$ and this completes the proof of the lemma.
\end{proof}

\begin{mytheorem}\label{thm:singlepathcorrect}
	Let $D = (V,E)$ be a graph and let $G =(N,\Sigma,P)$ be a grammar. Then for any $i, j$ and for any non-terminal $A \in N$, if $(A, l_A) \in a^{cf}_{i,j}$, then there is $i \pi j$, such that $A \xrightarrow{*} l(\pi)$ and the length of $\pi$ is equal to $l_A$.
\end{mytheorem}
\begin{proof}
	
	Since the matrix $a^{cf} = a^{(1)} \cup a^{(2)} \cup \cdots$, for any $i, j$ and for any non-terminal $A \in N$, if $(A, l_A) \in a^{cf}_{i,j}$, then there is $k \geq 1$, such that $A \in a^{(k)}_{i,j}$. By the lemma~\ref{lemma:singlepath}, if $(A, l_A) \in a^{(k)}_{i,j}$, then there is $i \pi j$, such that $A \xrightarrow{*} l(\pi)$ and the length of $\pi$ is equal to $l_A$. This completes the proof of the theorem.
\end{proof}

By the theorem~\ref{thm:correct}, we can determine whether $(i,j) \in R_A$ by asking whether $(A, l_A) \in a^{cf}_{i,j}$ for some $l_A$. By the theorem~\ref{thm:singlepathcorrect}, there is $i \pi j$, such that $A \xrightarrow{*} l(\pi)$ and the length of $\pi$ is equal to $l_A$. Therefore, we can find such a path $\pi$ of the length $l_A$ from the node $i$ to the node $j$ by a simple search. Thus, we show how the context-free path query evaluation using the single-path query semantics can be reduced to the calculation of matrix transitive closure $a^{cf}$. Note that the time complexity of the algorithm for context-free path querying w.r.t. the single-path semantics no longer depends on the Boolean matrix multiplications since we modify the matrix representation and operations on the matrix elements.
