\newpage
\section{Трансляцией программ с динамической памятью в Haskell\protect\footnotemark}\label{apdx:haskell}
\footnotetext{Эта трансляция проведена человеком, однако менее читаемая версия может быть получена полностью автоматически.}

% \begin{figure}
% \begin{minipage}[t]{.02\textwidth}
% \hfill
% \end{minipage}
\begin{minipage}[t]{.32\textwidth}
\begin{minted}[fontsize=\fontsize{6}{6},linenos=true,breaklines]{C}
#include <stdlib.h>
#include <stdio.h>

typedef struct node {
  int key;
  struct node *next;
} node_t;

void inc(node_t *node) {
  if (node) {
    node->key += 1;
    inc(node->next);
  }
}

void main() {
  node_t *a =
    malloc(sizeof(node_t));
  node_t *b =
    malloc(sizeof(node_t));
  node_t *c =
    malloc(sizeof(node_t));
  a->key = 10;
  b->key = 20;
  c->key = 30;
  a->next = b;
  b->next = c;
  c->next = NULL;
  inc(a);
  printf("%d\n", c->key);
}
\end{minted}
%\hfill
\end{minipage}
\begin{minipage}[t]{.6\textwidth}
\begin{minted}[fontsize=\fontsize{6}{6},breaklines]{haskell}
data Field =  Key | Next
type Addr = Int
type Loc = (Addr, Field)

ub _ = error "Undefined behaviour"

-- Heap before the recursive call of "inc"
h1_int :: (Loc -> Int) -> Addr -> Loc -> Int
h1_int ctx node loc@(addr, Key) | addr == node = ctx loc + 1
h1_int ctx _ loc = ctx loc

h1_ptr :: (Loc -> Addr) -> Loc -> Addr
h1_ptr ctx loc = ctx loc

-- The effect of "inc"
inc_int :: (Loc -> Int) -> (Loc -> Addr) -> Addr -> Loc -> Int
inc_int ctx_int ctx_ptr 0 loc = ctx_int loc
inc_int ctx_int ctx_ptr node loc =
    let next = ctx_ptr (node, Next) in
    inc_int (h1_int ctx_int node) (h1_ptr ctx_ptr) next loc

-- Heap before the call of "inc" from "main"
h2_int :: (Loc -> Int) -> Loc -> Int
h2_int _ (1, Key) = 10
h2_int _ (2, Key) = 20
h2_int _ (3, Key) = 30
h2_int ctx loc = ctx loc

h2_ptr :: (Loc -> Addr) -> Loc -> Addr
h2_ptr _ (1, Next) = 2
h2_ptr _ (2, Next) = 3
h2_ptr _ (3, Next) = 0
h2_ptr ctx loc = ctx loc

main =
    print $ show $ inc_int (h2_int ub) (h2_ptr ub) 1 (3, Key)
\end{minted}
\end{minipage}\\[3em]
Символьное исполнение без раскрутки даст $\compose{h_2}{inc}$ со следующим уравнением на состояние:
\[
\GRec{inc} = \GMerge{\pair{\mg{\li{node}}{0}}{\emptyheap}, \pair{\nmg{\li{node}}{0}}{\GCompose{h_1}{\GRec{inc}}}},
\]
где $h_1$ и $h_2$~--- это определённые кучи, полученные исполнением строчки~11~и~строк~17-28 соответственно.

Каждая определённая куча транслируется в две функции, по одной на соответствующий тип локации.
В результате программа транслируется композиционально: \texttt{inc} соответствует той же функции на Haskell, как и \texttt{main}; если появится ещё одна функция, вызывающая \texttt{inc}, готовая Haskell версия функции \texttt{inc} может быть переиспользована. В программах, полученных в результате такой трансляции, всегда будет только хвостовая рекурсия.