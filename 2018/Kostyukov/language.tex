\section{Демо язык}
\label{sec:language}

\begin{figure}
\begin{align*}
Program ::=\,&FunctionDecl^* \cr
FunctionDecl ::=\,&\funcid\,Block \cr
Expression :=\,&\mathbb{N} \mid \kwd{true} \mid \kwd{false} \mid \kwd{null} \mid nloc \mid \funcid \cr
				\mid\,&Expression\,BinOp\,Expression \mid UnOp\,Expression\, \cr
BinOp ::=\,&\texttt{+}\,|\,\texttt{-}\,|\,\texttt{==}\,|\,\texttt{!=}\,|\,\texttt{<}\,|\,\texttt{>}\,|\,\kwdand{}\,|\,\kwdor{} \cr
UnOp ::=\,&-\,|\,\kwdnot{} \cr
Block ::=\,&\{\,Statement^*\,\} \cr
Statement ::=\,&Block \cr
				\mid\,&\kwdnop{};\cr
				\mid\,&\kwdite{Expression}{Statement}{Statement} \cr
				\mid\,&nloc\,:=\,Expression; \cr
				\mid\,&nloc\,:=\,\kwdnew{}(Expression,\ldots,Expression); \cr
				\mid\,&(nloc_1,\ldots,nloc_n)\,:=\,\kwdread{(Expression)}; \cr
				\mid\,&\kwdwrite{}(Expression, Expression, \ldots, Expression); \cr
				\mid\,&\kwdfail{}; \cr
				\mid\,&\kwdcall{}\,FunctionRef; \cr
FunctionRef ::= &\funcid \mid nloc
\end{align*}
\caption{Синтаксис демо языка. Здесь $\funcid$ и $nloc$ являются идентификаторами в стиле языка C.}
\label{figure:syntax}
\end{figure}

Определим \ruquote{игрушечный} императивный язык программирования с операциями над кучей, для которого позже опишем композициональную процедуру верификации. Синтаксис языка показан на~\autoref{figure:syntax}.

Так как основной результат данной работы связан с верификацией программ с кучами, состояние программы на нашем языке будем описывать только кучей. По этой причине все функции не принимают аргументов и не имеют локальных переменных (они обычно хранятся на стеке), вместо них куча содержит набор \emph{именованных локаций} ($nloc$ на~\autoref{figure:syntax}). Так как функции также не должны возвращать значения (которые обычно тоже кладутся на стек), поведения функций полностью описываются изменениями в кучах и условиями пути в ошибочных ветках.

\sloppy{Именованные локации хранят \emph{примитивные значения}; другие локации хранят \emph{кортежи} примитивных значений. Примитивными значениями являются целые числа, имена функций и ссылки на локации в куче. Взаимодействие с кучей осуществляется при помощи инструкций \kwdnew{}, \kwdread{}, \kwdwrite{} и $nloc:=e$. Инструкция \mbox{$nloc:=$ \kwdnew($e_1,\ldots,e_n$)} выделяет кортеж $(e_1,\ldots,e_n)$ в куче и записывает ссылку на вновь созданную локацию в $nloc$. Инструкция \mbox{$(nloc_1, \ldots, nloc_n):=$ \kwdread{}($ref$)} записывает элементы кортежа, хранящегося по ссылке $ref$, в соответствующие локации $nloc_1, \ldots, nloc_n$. Мы допускаем написание спецсимвола ``\_'' вместо локаций $nloc_1, \ldots, nloc_n$~--- в таком случае соответствующий элемент кортежа игнорируется. Инструкция \mbox{\kwdwrite{}($r, e_1, \ldots, e_n$)} записывает кортеж $(e_1, \ldots, e_n)$ по локации $r$. Инструкции \kwdread{} и \kwdwrite{} бросают исключения при обращении к ссылке \kwd{null}. Инструкция же \mbox{$nloc:=e$} безопасна: она непосредственно перезаписывает содержимое локации $nloc$ значением $e$.}

Другие инструкции и выражения имеют стандартную семантику. В нашем языке отсутствует конструкция для построения циклов, однако допускаются рекурсивные функции. Заметим, что имена функций являются выражениями, а значит язык позволяет определять функции высшего порядка любой сложности.

Далее мы будем рассматривать только корректно типизированные программы: первые аргументы \kwdread{} и \kwdwrite{} должны вычисляться в ссылки; арности кортежей \kwdread{} и \kwdwrite{} также должны быть корректны; типы кортежа аргументов \kwdwrite{} должны соответствовать типам кортежа значений по записываемой ссылке; арифметические операторы не должны применяться к локациям и идентификаторам функций. Также мы предполагаем, что программы не читают из неинициализированных локаций, все функции, соответствующие их идентификаторам, определены, и множества функциональных идентификаторов и имён локаций не пересекаются. Наш демо язык не даёт возможности управлять памятью на низком уровне и освобождать выделенную память. Таким образом, есть всего два класса ошибок, которые необходимо находить: разыменования \kwd{null} и достижимость инструкции \kwdfail{}.


\begin{figure}
\begin{lstlisting}[style=demolang]
Inc {@\label{line:begin-inc}@
	if node != null then {
		(visited, value, next) := read(node);
		if visited == 0 then {
			write(node, 1, value + 1, next);
			node := next;
			call Inc;
		} else nop;
	} else nop;
}@\label{line:end-inc}@

Test {@\label{line:begin-test}@
	// node points to a head of linked list
	// initialized in caller
	if node != null then {
		(visited, oldValue, next) := read(node);
		write(node, 0, oldValue, next);
		call Inc;
		(_, newValue, _) := read(node);
		if newValue != oldValue + 1 then fail;@\label{line:throw}@
		else nop;
	} else nop;
}@\label{line:end-test}@
\end{lstlisting}
\caption{Программа, модифицирующая динамическую память.}
\label{figure:prog}
\end{figure}

\begin{exmp}\label{exmp:prog}
На \autoref{figure:prog} представлен пример программы на демо языке. Здесь определена рекурсивная функция \texttt{Inc}, которая обходит связный список, хранящийся по ссылке \texttt{node}, и инкрементирует хранящиеся в каждой вершине значения. Также определена функция \texttt{Test}, которая принимает связный список (в котором может быть цикл), проверяет его непустоту и вызывает \texttt{Inc}. Затем она проверяет, что значение в голове списка увеличилось на единицу. Таким образом, верификатор должен показать, что инструкция \kwdfail{} на строке~\ref{line:throw} недостижима.
\end{exmp}
